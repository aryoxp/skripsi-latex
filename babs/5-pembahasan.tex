\newpage
\chapter{Pembahasan}

Pembahasan berfungsi untuk menerjemahkan makna dari hasil yang diperoleh untuk menjawab pertanyaan atau masalah penelitian. Fungsi lainnya adalah untuk menjelaskan pemahaman baru yang didapatkan dari hasil penelitian, yang diharapkan berguna dalam pengembangan keilmuan. Dalam penelitian tingkat lanjut, fungsi pembahasan yang kedua ini sangat penting karena dapat menunjukkan kontribusi penulis terhadap pengembangan keilmuan. Akan tetapi, dalam penelitian tingkat skripsi, fungsi yang kedua ini dapat diterapkan secara terbatas karena pendidikan S1 tidak dituntut untuk pengembangan keilmuan secara substansial, tetapi cukup terhadap pemahaman personal dalam implementasi konsep atau teori. 

\section{Subbab Lima Satu}

Dalam menjawab masalah penelitian, penulis diminta untuk melakukan evaluasi kritis terhadap hasil yang diperoleh. Tergantung dari fokus penelitian, beberapa contoh pertanyaan kritis yang dapat dijawab adalah:
\begin{itemize}
  \item Seberapa jauh tujuan penelitian telah tercapai?
  \item Apakah aplikasi atau sistem yang dibangun sesuai dengan tujuannya?
  \item Apakah metode atau praktik perancangan dan implementasi yang baik telah dijalankan?
  \item Apakah teknologi implementasi yang tepat telah dipilih? Dan sebagainya. 
\end{itemize} 

\subsection{Subbab Lima Satu Satu}

Dalam menjelaskan pemahaman baru yang didapatkan, penulis dapat mengubungkan hasil penelitian dengan pengetahuan teoritik atau penelitian sebelumnya yang telah dibahas. Kaitan antara hasil penelitian dan pengetahuan teoritik misalnya berupa:
\begin{itemize}
  \item pendapat tentang metode yang digunakan dari pustaka, apakah dapat digunakan dengan baik secara langsung, dengan penyesuaian, atau dengan batasan tertentu;
  \item konfirmasi tentang batasan dari metodologi yang digunakan sehingga dapat berpengaruh pada hasil;
  \item penjelasan tentang informasi penting pada penelitian lainnya yang membantu penulis untuk menerjemahkan data penelitian penulis;
  \item penjelasan tentang kemungkinan hasil dari penelitian lainnya yang dapat dikombinasikan dengan penelitian penulis untuk memberikan pengetahuan baru; dan sebagainya. 
\end{itemize}

\subsection{Subbab Lima Satu Dua}
Penulis dapat merefleksikan apa yang telah dipelajari selama melakukan penelitian, tetapi harus tetap terfokus dengan masalah penelitian ini dan tidak melebar ke masalah lainnya. Hal-hal yang berada di luar fokus peneltian tetapi penting dan menarik untuk diteliti dapat disarankan sebagai bahan penelitian berikutnya. Hal ini dapat dipertegas di bab Kesimpulan/ Penutup. 

\section{Subbab Lima Dua}

Hasil dan pembahasan dapat diletakkan dengan kemungkinan berikut:
\begin{enumerate}
  \item Dipisahkan secara fisik ke dalam bab-bab yang berbeda
  \item Dipisahkan secara fisik ke dalam dua atau lebih paragraf atau subbab yang berbeda tetapi dalam bab yang sama
  \item Dileburkan menjadi satu dalam paragraf, dijelaskan secara naratif-deskriptif, terdistribusi ke satu atau lebih bab yang ada 
\end{enumerate}

\subsection{Subbab Lima Dua Satu}

Cara pertama atau kedua membantu pembaca yang ingin memisahkan observasi dan terjemahan dari observasi tersebut sehingga mereka dapat menilai kualitas dari masing-masing proses dengan lebih mudah. Kadang-kadang cara kedua lebih banyak dipilih daripada cara pertama jika data yang harus dipresentasikan yang cukup banyak dan laporan penelitian cukup panjang agar pembaca tidak perlu menunggu presentasi dari seluruh data selesai baru dapat membaca penerjemahannya. Cara pertama dan kedua ini banyak digunakan untuk penelitian yang bersifat kuantitatif, baik itu deskriptif, eksplanatori, maupun implementatif.    

\subsection{Subbab Lima Dua Dua}

Cara ketiga biasanya digunakan jika data, analisis, dan penafsirannya sulit dipisahkan. Pemisahannya terkadang justru membuat laporan penelitian sulit dibaca. Hal ini dapat berlaku pada tipe penelitian yang bersifat kualitatif, baik itu deskriptif ataupun analitik/eksplanatori. 

Pada dasarnya peletakan dan jumlah bab untuk hasil dan pembahasan sebaiknya disesuaikan karakter penelitian masing-masing. Judul bab pun tidak harus secara eksplisit "Hasil" dan "Pembahasan" tetapi dapat digantikan dengan nama yang lebih deskpritif dan tematik. 

\section{Subbab Lima Tiga}

Contoh struktur skripsi untuk implementatif pembangunan dan nonimplementatif eksperimental dapat dilihat pada kedua subbab berikut. 

\subsection{Contoh Struktur Penelitian Implementatif Pembangunan}

Berikut ini adalah contoh bab-bab yang terdapat pada penelitian implementatif pembangunan sistem perangkat lunak. 
\begin{displayquote}
  Bab 1 Pendahuluan \\
  Bab 2 Landasan Kepustakaan \\
  Bab 3 Metodologi Penelitian \\
  Bab 4 Kebutuhan \\
  Bab 5 Perancangan dan Implementasi \\
  Bab 6 Pengujian \\
  Bab 7 Penutup
\end{displayquote}
Bab 1 sampai Bab 3 memuat informasi yang sesuai dengan panduan sebelumnya. Isi dari bab-bab berikutnya: 

\begin{displayquote}
  Bab 4 Persyaratan:
  \begin{itemize}
    \item Pernyataan masalah (problem statement), yang lebih elaboratif daripada yang di Pendahuluan.
    \item Identifikasi pemangku kepentingan (stakeholders) dan aktor (actors) sistem.
    \item Daftar terstruktur persyaratan/kebutuhan perangkat lunak, secara fungsional, data, dan non-fungsional
    \item Use cases, use case diagrams, dan use case specifications, dan sebagainya. 
  \end{itemize} 
  Bab 5 Perancangan dan Implementasi:
  \begin{itemize}
    \item Rancangan arsitektur: deskripsi struktur dan setiap komponen utama  
    \item Representasi data dalam model data dan basis data 
    \item Detil implementasi dari fungsi-fungsi utama yang menjadi fokus
  \end{itemize}
  Bab 6 Pengujian dan Evaluasi
  \begin{itemize}
    \item Strategi, rencana, kasus, dan data pengujian
    \item Ringkasan hasil pengujian perangkat lunak, termasuk data dan analisisnya (detilnya di Lampiran)
    \item Evaluasi hasil proyek secara keseluruhan
  \end{itemize}
  Bab 7 Penutup
  \begin{itemize}
    \item Ringkasan dari capaian proyek
    \item Saran pengembangan lebih lanjut
  \end{itemize}
\end{displayquote}

Pada contoh struktur ini "hasil" tersebar di beberapa bab mulai Bab 4 Persyaratan sampai Bab 6, sedangkan "pembahasan" secara keseluruhan terhadap masalah penelitian terdapat di Bab 6. Yang dimaksud dengan pengujian dalam Bab 6 terfokus pada pengujian persyaratan perangkat lunak, sedangkan evaluasi berfungsi sebagai "pembahasan" secara keseluruhan, yaitu menentukan apakah "hasil" sudah menjawab masalah penelitian yang dirumuskan pada Bab 1. 

Sebagai catatan, Bab 3 Metodologi umumnya menjelaskan model proses perangkat lunak yang digunakan. Jika strategi untuk setiap aktivitasnya (analisis persyaratan, perancangan, dan seterusnya) sudah dijelaskan di Bab 3 ini juga, maka bab-bab lainnya yang berhubungan dengan aktivitas-aktivitas ini masing-masing langsung dapat menjelaskan hasil pelaksanaan metodenya. 

\subsection{Contoh Struktur Penelitian Nonimplementatif Eksperimental}

Berikut ini adalah contoh bab-bab yang terdapat pada penelitian implementatif pembangunan sistem perangkat lunak. 
\begin{displayquote}
  Bab 1 Pendahuluan \\
  Bab 2 Landasan Kepustakaan \\
  Bab 3 Metodologi Penelitian \\
  Bab 4 Hasil \\
  Bab 5 Pembahasan \\
  Bab 6 Penutup
\end{displayquote}

Isi dari setiap bab dapat menyesuaikan dengan panduan yang telah dijelaskan sebelumnya. Jika diperlukan, Bab 4 dapat digabungkan dengan Bab 5, menjadi Hasil dan Pembahasan. 

Struktur dasar ini cukup universal sehingga dapat digunakan juga untuk tipe-tipe penelitian lainnya, khususnya jika belum ada struktur lain yang lebih tematik dan cocok untuk penelitian yang bersangkutan.