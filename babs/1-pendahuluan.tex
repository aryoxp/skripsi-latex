\newpage
\pagenumbering{arabic}
\setcounter{page}{1}
\setstretch{1.15}


\chapter{Pendahuluan}

% \parindent=20pt
% \setlength{\parindent}{15pt}

Bagian utama skripsi terdiri dari beberapa komponen atau bab yang tersusun dengan alur yang logis. Pendahuluan merupakan komponen/bab pertama yang harus menjelaskan apa yang dikerjakan dalam skripsi dan mengapa ini dikerjakan. 

\section{Latar Belakang}

Bagian ini memuat penjelasan mengenai latar belakang munculnya ide sehingga penelitian ini dilakukan. Untuk mendapatkan masalah atau pertanyaan penelitian, penulis dapat melakukan inferensi dari fakta-fakta pendukung yang mungkin diperoleh dari literatur atau pengamatan. Penulis harus menjelaskan mengapa masalah yang diteliti dianggap penting dan menarik. Dapat juga diuraikan kedudukan masalah yang teliti ini dalam lingkup permasalahan yang lebih luas. Dalam menjelaskannya, penulis dapat menggunakan teknik piramida terbalik, yaitu memulai penjelasan dari yang lebih umum diikuti dengan yang semakin khusus dan terfokus pada masalah tertentu yang harus diselesaikan atau pertanyaan yang harus dijawab dalam penelitian ini. Dalam bagian ini dapat juga dimasukkan beberapa uraian singkat penelitian terdahulu yang dapat memperkuat alasan mengapa penelitian ini dilakukan. 

Untuk menjembatani antara latar belakang dan rumusan masalah, serta untuk membantu menjelaskan fokus penelitian, pada bagian akhir bagian ini dapat dituliskan sebuah pernyataan bahwa pengambilan topik skripsi didasarkan pada alasan yang telah dikemukakan, misalnya "Berdasarkan kebutuhan akan akurasi dari pengukuran kadar gula dalam darah diperlukan suatu perangkat lunak bantu yang akan dikembangkan dalam skripsi ini". Yang harus diperhatikan dalam penulisan latar belakang adalah adanya kesinambungan penjelasan antara latar belakang dengan bagian-bagian lain yang ditulis sesudahnya (rumusan masalah, tujuan, manfaat, dan batasan masalah).


\section{Rumusan Masalah}

Bagian ini memuat pertanyaan penelitian (research questions) yang dituliskan dalam kalimat tanya untuk mengarahkan penelitian, mendorong peneliti untuk menjawabnya, dan menarik minat pembaca. Pertanyaan penelitian umumnya memiliki ciri-ciri sebagai berikut:

\begin{enumerate}
  \item Jelas: disampaikan dengan struktur bahasa Indonesia yang baku, benar, dan mudah dipahami
  \item Relevan: sesuai dengan apa yang ingin diteliti dan menggunakan istilah-istilah yang sesuai dengan masalah serta konteks keilmuan terkait
  \item Fokus: terarah pada masalah yang ingin diselesaikan atau fenomena yang akan dijelaskan
  \item Menarik: diusahakan dapat mendorong keinginan peneliti untuk menjawab pertanyaan ini dan merangsang pembaca untuk mengikuti lebih jauh penelitian ini
  \item Dapat terjawab: dapat dijawab atau diukur hasilnya melaui proses penelitian sesuai dengan batasan waktu dan sumber daya yang ada 
\end{enumerate}


Berikut beberapa contoh pertanyaan penelitian yang sesuai dengan topik dan permasalahannya masing-masing:

\noindent\underline{Contoh 1:}

\noindent Topik: 

\begin{displayquote}
Pengembangan sistem pendukung keputusan seleksi penerimaan peserta didik baru menggunakan metode ELECTRE dan SAW (Studi kasus: SMA Brawijaya Smart School Kota Malang)
\end{displayquote}

Rangkuman masalah umum dari latar belakang (sudah tergambarkan dan tertuang dalam latar belakang dan tidak perlu dituliskan dalam subbab atau seksi tersendiri):

\begin{displayquote}
SMA BSS Malang memiliki kesulitan dalam proses seleksi penerimaan peserta didik baru berdasarkan keminatan masing-masing dengan mekanisme yang masih manual. Metode ELECTRE dan SAW dapat dimanfaatkan untuk mengolah data calon peserta didik dalam menentukan rekomendasi peserta didik baru yang diterima dalam kelompok peminatan tertentu.
\end{displayquote}

\noindent Pertanyaan penelitian:

\begin{enumerate}
  \item Bagaimanakah rancangan algoritme yang menggunakan metode ELECTRE dan SAW dalam sistem pendukung keputusan untuk seleksi penerimaan peserta didik baru SMA BSS Malang? 
  \item Bagaimanakah tingkat akurasi sistem pendukung keputusan Seleksi Penerimaan Peserta Didik Baru SMA BSS Kota Malang menggunakan metode ELECTRE dan SAW tersebut?  
\end{enumerate}

\noindent \underline{Contoh 2:}

\noindent Topik: 

\begin{displayquote}
Pengembangan sistem perangkat lunak untuk administrasi pendidikan di Pondok Pesantren Nurul Huda Malang
\end{displayquote}

Rangkuman masalah dari latar belakang (sudah tergambarkan dan tertuang dalam latar belakang dan tidak perlu dituliskan dalam subbab atau seksi tersendiri):

\begin{displayquote}
Pondok Pesantren Nurul Huda Malang membutuhkan sebuah sistem perangkat lunak yang dapat membantu pelaksanaan proses-proses bisnis di dalamnya, khususnya dalam administrasi pendidikan. Beberapa masalah ditemukan dalam proses-proses bisnis tersebut. Masalah ini diharapkan dapat terselesaikan dengan bantuan sejumlah fungsi yang ditawarkan oleh sistem perangkat lunak ini.
\end{displayquote} 

\noindent Pertanyaan penelitian:
\begin{enumerate}
  \item Bagaimanakah hasil analisis dan spesifikasi persyaratan sistem perangkat lunak untuk administrasi pendidikan di Pondok Pesantren Nurul Huda Malang yang sesuai dengan kebutuhan organisasi tersebut? 
  \item Bagaimanakah rancangan sistem perangkat lunak yang sesuai dengan spesifikasi persyaratan sistem tersebut? 
  \item Bagaimanakah hasil implementasi sistem perangkat lunak yang sesuai dengan rancangan sistem tersebut?
  \item Bagaimanakah hasil pengujian sistem perangkat lunak untuk administrasi pendidikan di pondok pesantren tersebut?
\end{enumerate}

\noindent\underline{Contoh 3:}

\noindent Topik: 

\begin{displayquote}
  Optimasi deteksi marker pada NyARToolKit menggunakan metode Ransac
\end{displayquote}

Rangkuman masalah dari latar belakang (sudah tergambarkan dan tertuang dalam latar belakang dan tidak perlu dituliskan dalam subbab atau seksi tersendiri):

\begin{displayquote}
Pembacaan marker pada aplikasi berbasis Augmented reality (AR) menggunakan pustaka NyARToolKit 4.0.3 masih kurang optimal jika digunakan untuk membaca marker yang tidak ideal. Untuk mengatasi kondisi tersebut, dibutuhkan metode, seperti metode Ransac, untuk mengoptimalkan kinerja aplikasi AR dalam membaca marker yang tidak ideal.
\end{displayquote}

\noindent Pertanyaan penelitian:

\begin{enumerate}
  \item Bagaimanakah rancangan aplikasi yang dapat meningkatkan kinerja AR terhadap pengenalan marker tidak ideal dengan metode RANSAC?
  \item Bagaimanakah mengimplementasikan algoritma metode RANSAC pada pustaka NyARToolKit 4.0.3?
  \item Bagaimana pengaruh metode RANSAC terhadap peningkatan performa pendeteksian marker? 
\end{enumerate}

\noindent\underline{Contoh 4:}

\noindent Topik: 
\begin{displayquote}
  Pengujian usability desain tata letak papan ketik berbasis QWERTY untuk penulisan teks Arab (studi kasus: Intellark, Nonosoft Khot, dan Arabic Pad)
\end{displayquote}


Rangkuman masalah dari latar belakang (sudah tergambarkan dan tertuang dalam latar belakang dan tidak perlu dituliskan dalam subbab atau seksi tersendiri):

\begin{displayquote}
Intellark, Nonosoft Khot, dan Arabic Pad adalah desain tata letak papan ketik berbasis QWERTY untuk penulisan teks Arab yang memiliki karakter masing-masing. Sampai sejauh ini belum diketahui tingkat usability ketiga desain tersebut terhadap pengguna Indonesia, khususnya dalam aspek kecepatan pengetikan, tingkat kesalahan pengetikan, dan kemudahan untuk dipelajari.  
\end{displayquote}

\noindent Pertanyaan penelitian:
\begin{displayquote}
  Bagaimana perbandingan tingkat usability dari Intellark, Nonosoft Khot, dan Arabic Pad dalam menuliskan teks Arab untuk pengguna Indonesia, dalam aspek:
  \begin{enumerate}
    \item kecepatan pengetikan,
    \item tingkat kesalahan pengetikan, dan
    \item kemudahan untuk dipelajarinya?   
  \end{enumerate}
\end{displayquote}

\noindent\underline{Contoh 5:}

\noindent Topik:

\begin{displayquote}
  Pengaruh kepercayaan pelanggan terhadap tingkat retensi pelanggan di Gerai XXX
\end{displayquote}

\noindent Pertanyaan penelitian:

\begin{enumerate}
  \item Bagaimana hubungan kepercayaan pelanggan terhadap tingkat retensi pelanggan di Gerai XXX?
  \item Bagaimana pengaruh kepercayaan pelanggan terhadap tingkat retensi pelanggan di Gerai XXX?  
\end{enumerate}

\noindent Catatan: 

Ada yang berpendapat bahwa rumusan masalah berisi pernyataan masalah (problem statement) sebagai rangkuman dari masalah yang tertuang dalam latar belakang. Untuk menghindari kerancuan, dalam panduan skripsi ini rumusan masalah diartikan sebagai pertanyaan penelitian (bukan pernyataan masalah) dengan defnisi, ciri-ciri, dan contoh tersebut sebelumnya. 

Jika terdapat hipotesis yang harus diuji, hipotesis dapat dituliskan pada seksi rumusan masalah ini dengan kalimat pernyataan yang sederhana, spesifik dan jelas, menyebutkan variabel-variabel yang diuji. Hipotesis dapat juga dituliskan dalam bagian terpisah "Rumusan hipotesis" dan diletakkan setelah rumusan masalah. Hipotesis merupakan dugaan atau jawaban sementara dari pertanyaan atau masalah penelitian yang masih harus dibuktikan kebenarannya dalam penelitian ini.

Contoh hipotesis untuk topik dan pertanyaan penelitian pada Contoh 5 sebelumnya:
\begin{enumerate}
  \item Terdapat hubungan positif antara kepercayaan pelanggan dan tingkat retensi pelanggan di Gerai XXX. 
  \item Terdapat pengaruh positif antara kepercayaan pelanggan dan tingkat retensi pelanggan di Gerai XXX.
\end{enumerate}

\section{Tujuan}

Bagian ini berisi tujuan yang ingin dicapai dari skripsi ini. Tujuan yang ditulis harus dapat memberikan arah pada capaian penelitian. Tujuan ini dapat terdiri dari beberapa butir yang masing-masing harus dituliskan dalam kalimat pernyataan yang sederhana dan jelas, sesuai dengan masalah penelitian dan hasil yang ingin dicapai. 

Berikut ini beberapa contoh penulisan tujuan sesuai dengan contoh-contoh rumusan masalah pada seksi sebelumnya.

\noindent\underline{Contoh 1:}

\noindent Tujuan:
\begin{enumerate}
  \item Merancang algoritme untuk seleksi penerimaan penerimaan peserta didik baru SMA BSS Malang dengan metode ELECTRE dan SAW ke dalam sebuah sistem pendukung keputusan
  \item Menguji tingkat akurasi sistem pendukung keputusan Seleksi Penerimaan Peserta Didik Baru SMA BSS Kota Malang yang menggunakan metode ELECTRE dan SAW 
\end{enumerate}

\noindent\underline{Contoh 2:}

\noindent Tujuan:
\begin{enumerate}
  \item Menganalisis dan menyusun spesifikasi persyaratan sistem perangkat lunak untuk administrasi pendidikan di Pondok Pesantren Nurul Huda Malang
  \item Merancang sistem perangkat lunak sesuai persyaratan untuk sistem perangkat lunak tersebut
  \item Mengimplementasikan rancangan sistem perangkat lunak tersebut
  \item Menguji sistem perangkat lunak tersebut secara fungsional dan non-fungsional (sesuai kebutuhan/masalah yang difokuskan)
\end{enumerate}

\noindent\underline{Contoh 3:}

\noindent Tujuan:
\begin{enumerate}
  \item Merancang aplikasi yang dapat meningkat kinerja AR terhadap marker yang tidak ideal dengan metode RANSAC
  \item Mengimplementasikan algoritma metode RANSAC pada pustaka NyARToolKit 4.0.3
  \item Menilai pengaruh metode RANSAC terhadap peningkatan performa marker. 
\end{enumerate}

\noindent\underline{Contoh 4:}

\noindent Tujuan:

Mengevaluasi usability dan mengetahui perbandingan tingkat usability dari Intellark, Nonosoft Khot, dan Arabic Pad dalam menuliskan teks Arab untuk pengguna Indonesia, khususnya dalam aspek:
\begin{enumerate}
  \item kecepatan pengetikan,
  \item tingkat kesalahan pengetikan,
  \item dan kemudahan untuk dipelajarinya  
\end{enumerate}

\noindent\underline{Contoh 5:}

\noindent Tujuan:
\begin{enumerate}
  \item Mengetahui hubungan kepercayaan pelanggan terhadap tingkat retensi pelanggan di Gerai XXX.
  \item Mengetahui pengaruh kepercayaan pelanggan terhadap tingkat retensi pelanggan di Gerai XXX.
\end{enumerate}

Tujuan penelitian dapat juga dituliskan terdiri dari tujuan umum (aim) dan tujuan-tujuan khusus (objectives) yang mengelaborasi tujuan umumnya. Contohnya adalah:

\begin{displayquote}
  Tujuan umum:

  Mengembangkan aplikasi piranti bergerak eHalal untuk identifikasi produk halal MUI di supermarket 
  
  Tujuan khusus:
  \begin{enumerate}
    \item Mengidentifikasi persyaratan fungsional dan non fungsional aplikasi eHalal
    \item Merancang aplikasi eHalal dengan pemodelan berorientesi objek
    \item Mengimplementasikan aplikasi eHalal dengan teknologi berorientasi obyek
    \item Menguji aplikasi eHalal sesuai dengan persyaratan fungsional dan non fungsionalnya 
  \end{enumerate}
\end{displayquote}
  
Sebagai tambahan, jika sebuah penelitian dimaksudkan untuk menguji hipotesis, maka paling tidak salah satu tujuannya berhubungan dengan pengujian hipotesis tersebut.


\section{Manfaat}

Manfaat penelitian dapat diuraikan sebagai dampak atau konsekuensi positif penelitian terhadap ruang lingkup masalah yang lebih luas dan/atau terhadap para pemangku kepentingan (stakeholders) yang terlibat di dalamnya. Manfaat penelitian seharusnya tidak meliputi pernyataan "untuk memenuhi persyaratan mencapai gelar sarjana" di program studi yang bersangkutan karena ini merupakan persyaratan akademik dan administratif  institusi, tidak berhubungan dengan substansi penelitiannya.

\section{Batasan Masalah}

Bagian ini dapat dituliskan untuk membantu menjelaskan ruang lingkup masalah penelitian dengan menyatakan hal-hal yang menjadi batasan dan asumsi-asumsi yang digunakan untuk menyelesaikan masalah yang sudah dirumuskan. 

Batasan-batasan yang sangat teknis dan tidak langsung berhubungan dengan fokus masalahnya, jika tetap diperlukan, sebaiknya diletakkan di bab lain yang lebih relevan. Sebagai contoh, untuk meneliti implementasi algoritma tertentu ke dalam sebuah kasus dengan fokus akurasi algoritme, jenis aplikasi editor untuk penyusunan kode program tidak perlu dituliskan di batasan masalah, tetapi lebih tepat di bab metodologi atau implementasi.   

Bagian batasan masalah ini dapat dihilangkan jika ruang lingkup masalah yang diuraikan dan direfleksikan melalui latar belakang, rumusan masalah, dan tujuan penelitian sudah cukup jelas.

\section{Sistematika Pembahasan}

Bagian ini berisi struktur skripsi ini mulai Bab Pendahuluan sampai Bab Penutup dan deskripsi singkat dari masing-masing bab. Diharapkan bagian ini dapat membantu pembaca dalam memahami sistematika pembahasan isi dalam skripsi ini. 